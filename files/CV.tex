\documentclass[11pt,a4paper,sans]{moderncv}        % possible options include font size ('10pt', '11pt' and '12pt'), paper size ('a4paper', 'letterpaper', 'a5paper', 'legalpaper', 'executivepaper' and 'landscape') and font family ('sans' and 'roman')
\usepackage{amssymb}
\moderncvstyle{classic}                    
\moderncvcolor{blue}                         
\usepackage[utf8]{inputenc}     
\usepackage{lmodern}       
\usepackage{anyfontsize}

\usepackage[scale=0.85]{geometry}
\usepackage{enumitem}
\setlist[itemize]{itemsep=0mm}
\newcommand{\da}{$^\dagger$}

% personal data
\name{Narendra}{Mukherjee}
\title{}
%\address{}{Yale University Chemistry Department}{New Haven, 06511}
%\phone[mobile]{+1~(203)~668~3568}
%\social[linkedin][www.linkedin.com/in/sbsinha]{linkedin.com/in/sbsinha}
\homepage{narendramukherjee.github.io}
\email{narendra@brandeis.edu}                               % optional, remove / comment the line if not wanted
\begin{document}
\makecvtitle

\vspace{-0.5in}

\section{Education}
\cventry{March 2019}{Ph.D. in Neuroscience}{Brandeis University}{Waltham, MA}{\newline {\underline{Dissertation title:} {Behaviorally relevant sensory cortical population dynamics in the rodent taste system}}}{\textbullet{HHMI International Predoctoral Fellow (2014-2017)}}

\cventry{May 2012}{Integrated BS-MS in Biological Sciences}{Indian Institute of Science Education and Research}{Kolkata, India}{\newline \underline{Dissertation title:} {Optimality and Courtship Behaviour in Zebrafish, \textit{Danio Rerio}}}{\textbullet{Awarded Director's Gold Medal (best performing student in Biological Sciences)}}

\section{Publications/Work in Progress}
\cvitem{2019}{{Levitan D., Lin J., Wachutka J., \textbf{Mukherjee N.,} Nelson S.B., Katz D.B. \textit{Single neuron and population coding of taste in the gustatory cortex of awake mice.} \emph{\textbf{bioRxiv. }}{\textbf{doi:} doi.org/10.1101/575522}}}

\cvitem{2018}{{\textbf{Mukherjee N.,} Wachutka J., Katz D.B. \textit{Dynamical structure of cortical taste responses revealed by precisely-timed optogenetic perturbation.} \emph{\textbf{bioRxiv. }}{\textbf{doi:} doi.org/10.1101/486043}}}

\cvitem{2018}{{Flores V.F, Parmet T., \textbf{Mukherjee N.,} Nelson S., Levitan D., Katz D.B. \textit{The role of the gustatory cortex in incidental experience-evoked enhancement of later taste learning.} \emph{\textbf{Learning and Memory. }}{\textbf{25(11): }587 - 600}}}

\cvitem{2017}{{\textbf{Mukherjee N.,} Wachutka J., Katz D.B. \textit{Python meets systems neuroscience: affordable, scalable and open-source electrophysiology in awake, behaving rodents.} \emph{\textbf{Proceedings of the 16th Python in Science Conference. }}{97 - 104}}}

\cvitem{2016}{{Sadacca B.F., \textbf{Mukherjee N.,} Vladusich T., Li J.X., Katz, D.B., Miller P. \textit{The Behavioral Relevance of Cortical Neural Ensemble Responses Emerges
Suddenly.} \emph{\textbf{Journal of Neuroscience. }}{\textbf{36(3): }655 - 669}}}

\cvitem{2013}{{Varma V., \textbf{Mukherjee N.,} Nisha N.K., Sharma V.K. \textit{Strong (Type 0) phase resetting of activity/rest rhythm of fruit flies, Drosophila melanogaster, at low temperature.} \emph{\textbf{Journal of Biological Rhythms. }}{\textbf{28(6): }380 - 389}}}

\cvitem{2012}{{Nisha N.K., \textbf{Mukherjee N.,} Sharma V.K. \textit{Robustness of circadian timing systems evolves in fruit flies Drosophila melanogaster as a correlated response to selection for adult emergence in a narrow window of time.} \emph{\textbf{Chronobiology International. }}{\textbf{29(10): }1312 - 1328}}}

\cvitem{2012}{{\textbf{Mukherjee N.,} Nisha N.K., Yadav P., Sharma V.K. \textit{A model based on oscillatory threshold and build up of a developmental substance can explain gating of adult emergence in fruit flies D. melanogaster.} \emph{\textbf{Journal of Experimental Biology. }}{\textbf{215(17): }2960 - 2968}}}

\section{Invited Talks}
\cvitem{2018}{\textbf{Discrete cortical population activity states underlie taste processing and consumption behavior.} \textit{Neuroscience Statistics Research Laboratory, Massachusetts Institute of Technology (MIT), Cambridge, MA}}

\cvitem{2018}{\textbf{Openness in Science and Society.} \textit{Indian Institute of Science Education and Research, Kolkata, India}}

\cvitem{2017}{\textbf{Building affordable, scalable and open-source tools to study behaviorally relevant neural population dynamics.} \textit{Center for Depression, Anxiety and Stress Research, McLean Hospital, Belmont, MA}}

\cvitem{2017}{\textbf{Systems neuroscience with Python: peering into the "black box".} \textit{Boston Python Meetup Group, Cambridge, MA}}
 
\section{Selected Poster Presentations}
\cvitem{2018}{{\textbf{Mukherjee N.,} Wachutka J., Katz D.B. \textit{Dynamical structure of cortical taste responses revealed by precisely-timed optogenetic perturbation.} \emph{\textbf{Computational and Systems Neuroscience (Cosyne) 2018, Denver, CO}}}}

\cvitem{2017}{{\textbf{Mukherjee N.,} Wachutka J., Katz D.B. \textit{Optogenetically perturbing behaviorally relevant stochastic cortical population dynamics.} \emph{\textbf{Statistical Analysis of Neuronal Data (SAND8) at Pittsburgh, PA}}}}

\cvitem{2016}{{\textbf{Mukherjee N.,} Wachutka J., Katz D.B. \textit{Perturbing behaviorally relevant cortical population activity states.} \emph{\textbf{Annual Meeting of the Society for Neuroscience (SfN) at San Diego, CA}}}}

\cvitem{2014}{{\textbf{Mukherjee N.,} Li J.X., Katz D.B. \textit{Ensemble dynamics in the rat gustatory cortex can precisely predict taste ingestion-rejection decisions.} \emph{\textbf{Annual Meeting of the Society for Neuroscience (SfN) at Washington, DC}}}}

\cvitem{2014}{{\textbf{Mukherjee N.,} Li J.X., Katz D.B. \textit{Ensemble dynamics in the rat gustatory cortex can precisely predict taste ingestion-rejection decisions.} \emph{\textbf{36th Annual Meeting of the Association for Chemoreception Sciences (AChemS) at Bonita Springs, FL}}}}

\section{Grants and Awards}
\cvitem{2017-2019}{\$29,513  (estimated) towards cloud computing resources on the Jetstream supercomputer of the Extreme Science and Engineering Discovery Environment (XSEDE) of the National Science Foundation (NSF) (as administrator).}

\cvitem{2014-2017}{\$70,000 per year towards tuition and fellowship from the Howard Hughes Medical Institute (HHMI) as part of the International Predoctoral Fellowship.}

\cvitem{2014}{Pulin Sampat Memorial Award for the Best Teaching Fellow in the Life Sciences,
Brandeis University.}

\cvitem{2008-2012}{Innovation in Science Pursuit for Inspired Research (INSPIRE) Scholarship for Higher Education (SHE), DST, Govt. of India.}

\cvitem{2012}{Nominated for the Dr. Shyama Prasad Mukherjee (SPM) Fellowship, CSIR, Govt.
of India.}

\cvitem{2010, 2011}{Summer Research Fellowship, Jawaharlal Nehru Centre for Advanced Scientific Research (JNCASR), Bangalore, India.}

\cvitem{2010}{Rajiv Gandhi Science Talent Research Scholarship, Rajiv Gandhi Foundation, New
Delhi and JNCASR (Best project under Summer Research Fellowship, 2010).}

\cvitem{2010}{Best participant in SERC school in chronobiology 2010, Department of Science
and Technology (DST), Govt. of India.}

\cvitem{2008, 2009}{CSIR Program for Youth on Leadership in Science (CPYLS) associateship at Centre for Cellular and Molecular Biology (CCMB), Hyderabad, CSIR, Govt. of India.}

\cvitem{2008}{CNR Rao Education Foundation Prize, IISER Kolkata.}

%\section{Research Experience}
%\cvitem{2012-Present}{PhD dissertation work with Prof. Donald B. Katz, Brandeis University, Waltham, MA}{Using the rat taste system as a model, I study sensory processing and its mapping onto behavior from a dynamical systems perspective. My work demonstrates that inter-trial variability in the responses of taste cortex neurons, previously averaged across and dismissed as "noise", correlates strongly with the variability in the timing of the animal's swallow-expel decision. I am currently involved in testing the functional basis of this correlation by optogenetically perturbing specific epochs of firing in the taste cortex and examining its effect on swallow-expel behavior. To this end, I build probabilistic graphical models (primarily HMMs) to parse states of population activity in taste cortex neural ensembles and use precisely timed optogenetics to perturb behavior-related population activity before/after it emerges. These experiments are starting to provide evidence for a distributed sensory-motor attractor network in taste processing, characterized by stochastic shifts into a behaviorally-relevant stable state.}
%
%\cvitem{2011-2012}{Masters dissertation work with Prof. Anuradha Bhat, Indian Institute of Science Education and Research, Kolkata, India}{I studied male courtship using an economics inspired approach based on the law of diminishing returns. As the eggs laid by a female are increasingly fertilized by the sperms released by males, the rate of fertilization of the remaining eggs progressively goes down as postulated by the law of diminishing returns. I tested the predictions from this theoretical framework using zebrafish \textit{Danio rerio.} 


\section{Teaching Experience}
\cventry{2016}{NPSY 18a: Introduction to Learning and Behavior}{Brandeis University, Waltham, MA}{\newline Guest lecturer for section on Machine Learning and Artificial Intelligence}{}{}{}

\cventry{2016}{BIO 107a: Data Analysis and Statistics Workshop}{Brandeis University, Waltham, MA}{\newline Teaching Fellow with Prof. Steve Van Hooser}{}{}{}

\cventry{2014}{NBIO 136b: Computational Neuroscience}{Brandeis University, Waltham, MA}{\newline Teaching Fellow with Prof. Paul Miller}{}{}{}

\cventry{2013}{NBIO 45a: The Cognitive and Neurobiological Basis of Memory}{Brandeis University, Waltham, MA}{\newline Teaching Fellow with Prof. John Lisman}{}{}{}

\section{Professional Experience}
\cvitem{Reviewer}{Scipy 2018 Program Committee}

\cvitem{Reviewer}{An Introductory Course in Computational Neuroscience by Paul Miller, Brandeis University (MIT Press, forthcoming)}

\section{Technical Expertise}
\cvitem{Experimental}{Stereotactic rodent surgeries, chronic implantation of multielectrode bundles, simultaneous electrophysiology and optogenetics in awake rodents.}

\cvitem{Hardware}{Extensive experience with boards like the Raspberry Pi and Arduino. Built a low-cost, modular, open-source rodent electrophysiology, optogenetics and behavior system with the Raspberry Pi and amplifier chips from Intan Technologies as part of PhD thesis work.}

\cvitem{Software}{Python and Linux (expert), R and MATLAB (intermediate). Wrote spike sorting and analysis software for the electrophysiology system in Python. Experience with variational Bayesian methods and deep neural networks in \textit{Tensorflow}. Contributor to \textit{pymc3} and \textit{datashader}. Extensive experience working with HPC environments at Brandeis and at XSEDE (Jetstream).}

\section{References}
\cvitem{}{\textbf{Donald B Katz}{\newline{Professor of Psychology, Brandeis University}{\newline\underline{dbkatz@brandeis.edu}}}}
\vspace{5pt}
\cvitem{}{\textbf{Eve Marder} \newline{Victor and Gwendolyn Beinfield Professor of Neuroscience, Brandeis University} {\newline \underline {marder@brandeis.edu}}}
\vspace{5pt}
\cvitem{}{\textbf{Leslie C Griffith} \newline{Professor of Biology, and Director of the Volen National Center for Complex Systems, Brandeis University} {\newline \underline {griffith@brandeis.edu}}}
\vspace{5pt}
\cvitem{}{\textbf{Shantanu Jadhav} \newline{Assistant Professor of Psychology, Brandeis University} {\newline \underline {shantanu@brandeis.edu}}}
\vspace{5pt}
\cvitem{}{\textbf{Paul Miller} \newline{Associate Professor of Biology and Computational Neuroscience, Brandeis University} {\newline \underline {pmiller@brandeis.edu}}}

\end{document}